\chapter{Anhang}
\label{chap:anhang}

\section{Quellcode}
\label{sec:quellcode}

Der vollständige Quellcode des entwickelten Prototyps wird dieser Arbeit als separate Datei beigefügt. Die Codebasis umfasst folgende Hauptkomponenten:

\begin{itemize}
    \item \textbf{core/} -- Kernmodule für Objekterkennung, OCR und Linking
    \begin{itemize}
        \item \texttt{yolo\_detection.py} -- YOLO-Detektionslogik
        \item \texttt{ocr\_engine.py} -- Multi-Engine OCR-Kaskade
        \item \texttt{linking.py} -- Räumlicher Linking-Algorithmus
        \item \texttt{pipelineworker.py} -- Pipeline-Orchestrierung
        \item \texttt{image\_processing.py} -- Bildvorverarbeitung
    \end{itemize}
    \item \textbf{ui/} -- PyQt5-basierte Desktop-Anwendung (18 Module)
    \item \textbf{pdfcomparison/} -- Vergleichsmodul für Planversionen
    \item \textbf{uservalidation/} -- Human-in-the-Loop Validierungskomponenten
    \item \textbf{utils/} -- Hilfsfunktionen (DPI, Rotation, UUID)
    \item \texttt{config.py} -- Zentrale Konfiguration und Klassenparameter
    \item \texttt{database.py} -- Datenbankanbindung
    \item \texttt{main.py} -- Applikations-Einstiegspunkt
\end{itemize}

\vspace{1em}
\noindent\textbf{Bereitstellung:} Der Quellcode ist im folgenden Ordner dieser Arbeit beigefügt:\\[0.5em]
\texttt{2026\_MA Swain\_KIPrototypErkennungGleisplan\_Anhang\_Code}

