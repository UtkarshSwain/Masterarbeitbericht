\chapter{Anforderungsanalyse}
\label{chap:anforderungen}
Die in Kapitel~\ref{chap:theoretischeundtechnischegrundlagen} dargelegten theoretischen und technischen Grundlagen bilden das Fundament für die praktische Umsetzung des Prototyps. In den nachfolgenden Schritten erfolgt die Konkretisierung der Anforderungen an den zu entwickelnden Prototyp. Die vorliegende Analyse leitet sich aus den funktionalen Zielen des Projektes sowie den nicht-funktionalen Rahmenbedingungen im industriellen Umfeld der Siemens Mobility GmbH ab. Zu diesem Zweck werden die Systemgrenzen, Datenformate sowie potenzielle technische Herausforderungen strukturiert analysiert.

\section{Funktionale Anforderungen}
\label{sec:funktionaleanforderungen}
Im Rahmen dieser Arbeit wird ein Prototyp entwickelt, der die automatisierte Extraktion und Interpretation von Informationen aus technischen Gleisplänen ermöglicht. Die funktionalen Anforderungen (FA-001 bis FA-014) ergeben sich aus den spezifischen Aufgabenstellungen der Signaltechnik-Planung. Der Prototyp soll folgende Kernfunktionen erfüllen:

\subsection{Symbolerkennung und Objektklassifizierung}
\label{subsec:req_symbol}
Eine zentrale Anforderung ist die zuverlässige Detektion bahntechnischer Symbole in Vektor- oder Rastergrafiken.

\begin{itemize}
    \item \textbf{FA-001}\label{req:FA-001} \textbf{Erkennungsrate:} Der Prototyp \textbf{muss} mindestens 90\,\% der definierten Symbolklassen \textbf{detektieren} (gemessen als Recall des YOLO-Modells).

    \item \textbf{FA-002}\label{req:FA-002} \textbf{Rotationsinvarianz:} Wenn Symbole in beliebigen Winkeln ($0^\circ$ bis $360^\circ$) orientiert sind, \textbf{muss} der Prototyp in der Lage sein, diese Objekte korrekt zu \textbf{klassifizieren} und deren Rotationswinkel mit einer Genauigkeit von $\pm 5^\circ$ zu \textbf{bestimmen}.

    \item \textbf{FA-003}\label{req:FA-003} \textbf{Detektionsziele:} Der Prototyp \textbf{muss} 13 Symbolklassen detektieren (Tabelle~\ref{tab:symbolklassen}). Die Klasse \textit{coordinate} liefert Positionsinformationen (Kilometrierung), die den jeweiligen anderen Klassen zugeordnet werden. Alle 13 Klassen werden in Kapitel~\ref{chap:evaluation} quantitativ evaluiert.
\end{itemize}
\subsubsection{Symbolklassen}

Die folgenden 13 Objektklassen bilden die Detektionsziele des Prototyps. Die Klasse \textit{coordinate} nimmt dabei eine besondere Rolle ein: Sie liefert die Positionsinformation (Kilometrierung), die den anderen 12 Klassen zugeordnet wird:

\begin{longtable}{|p{5cm}|p{7.5cm}|}
    \hline
    \textbf{Objektklasse} & \textbf{Symbol} \\
    \hline
    \endhead

    Koordinate (Positionsangabe) &
    \parbox[c]{7cm}{\centering
        \vspace{0.2cm}
        \includegraphics[width=2cm]{images/symbols/Position.png}
        \vspace{0.2cm}
    } \\
    \hline

    Signal &
    \parbox[c]{7cm}{\centering
        \vspace{0.2cm}
        \includegraphics[width=2cm]{images/symbols/Signal2.png}
        \vspace{0.2cm}
    } \\
    \hline

    GKS (festkodiert) &
    \parbox[c]{7cm}{\centering
        \vspace{0.2cm}
        \includegraphics[width=2cm]{images/symbols/GKS1.png}
        \vspace{0.2cm}
    } \\
    \hline

    GKS (gesteuert) &
    \parbox[c]{7cm}{\centering
        \vspace{0.2cm}
        \includegraphics[width=2cm]{images/symbols/GKS2.png}
        \vspace{0.2cm}
    } \\
    \hline

    Gleismagnet &
    \parbox[c]{7cm}{\centering
        \vspace{0.2cm}
        \includegraphics[width=2cm]{images/symbols/gleismagnet.png}
        \vspace{0.2cm}
    } \\
    \hline

    Weichenblock &
    \parbox[c]{7cm}{\centering
        \vspace{0.2cm}
        \includegraphics[width=3cm]{images/symbols/Weiche2.png}
        \vspace{0.2cm}
    } \\
    \hline

    Haltepunkt &
    \parbox[c]{7cm}{\centering
        \vspace{0.2cm}
        \includegraphics[width=2cm]{images/symbols/haltepunkt3.png}
        \vspace{0.2cm}
    } \\
    \hline

    Isolierstoß &
    \parbox[c]{7cm}{\centering
        \vspace{0.2cm}
        \includegraphics[width=1cm]{images/symbols/Isolierstoߧ.png}
        \vspace{0.2cm}
    } \\
    \hline

    S-Verbinder &
    \parbox[c]{7cm}{\centering
        \vspace{0.2cm}
        \includegraphics[width=3cm]{images/symbols/SVERBINDER.png}
        \vspace{0.2cm}
    } \\
    \hline

    Prellbock &
    \parbox[c]{7cm}{\centering
        \vspace{0.2cm}
        \includegraphics[width=2cm]{images/symbols/prellblock.png}
        \vspace{0.2cm}
    } \\
    \hline

    Haltetafel &
    \parbox[c]{7cm}{\centering
        \vspace{0.2cm}
        \includegraphics[width=3cm]{images/symbols/Haltetafel.png}
        \vspace{0.2cm}
    } \\
    \hline

    Ende Weichen &
    \parbox[c]{7cm}{\centering
        \vspace{0.2cm}
        \includegraphics[width=3cm]{images/symbols/weichenende.png}
        \vspace{0.2cm}
    } \\
    \hline

    Weichengruppenende &
    \parbox[c]{7cm}{\centering
        \vspace{0.2cm}
        \includegraphics[width=3cm]{images/symbols/weichengruppeende.png}
        \vspace{0.2cm}
    } \\
    \hline

    \caption{13 Symbolklassen für die Datenextraktion}
    \label{tab:symbolklassen}
\end{longtable}

\subsection{Texterkennung und OCR-Integration}
\begin{itemize}
    \item \textbf{FA-004}\label{req:FA-004} \textbf{OCR-Genauigkeit:} Der Prototyp \textbf{muss} Textinformationen (Signalbezeichnungen, Kilometrierungen) zuverlässig \textbf{extrahieren}.

    \item \textbf{FA-005}\label{req:FA-005} \textbf{Robustheit:} Wenn Textregionen unter erschwerten Bedingungen vorliegen (niedriger Kontrast, Bildrauschen, Rotation $0^\circ$--$360^\circ$, Überlagerungen), \textbf{muss} der Prototyp in der Lage sein, diese Texte zuverlässig zu \textbf{extrahieren} (siehe Abbildung~\ref{fig:Ocrsample}).
    \begin{figure}[H]
    \centering
    \includegraphics[width=5cm]{images/Kapitel4/OCRsample.png} 
    \caption{Beispielhafte Extraktion von Text und Position am Symbol GKS}
    \label{fig:Ocrsample}
    \end{figure}
\end{itemize}

\subsection{Semantische Verknüpfung und Logik}

\begin{itemize}
    \item \textbf{FA-006}\label{req:FA-006} \textbf{Fahrtrichtungsdetektion:} Wenn ein Signal detektiert wird, \textbf{soll} der Prototyp die Wirkrichtung (steigend/fallend) basierend auf dem Rotationswinkel des Symbols \textbf{ableiten}.

    \item \textbf{FA-007}\label{req:FA-007} \textbf{Symbol-Koordinaten-Verknüpfung:} Der Prototyp \textbf{muss} jedes erkannte Gleisplanelement (Signal, GKS, GM-Block) automatisch mit der räumlich nächstgelegenen Kilometrierungsangabe mittels geometrischer Proximity-Analyse unter Berücksichtigung der Symbolorientierung \textbf{verknüpfen}.

    \item \textbf{FA-008}\label{req:FA-008} \textbf{Manuelle Korrektur:} Der Prototyp \textbf{soll} eine Funktion zur manuellen Überschreibung automatisch erstellter Symbol-Text-Verknüpfungen \textbf{bereitstellen}.
\end{itemize}

\subsection{Datenaufbereitung und Export}
\begin{itemize}
    \item \textbf{FA-009}\label{req:FA-009} \textbf{Excel-Export:} Der Prototyp \textbf{muss} extrahierte Daten vollautomatisch in vordefinierte Excel-Tabellen (.xlsx) mit korrekter Zuordnung zu Zeilen und Spalten \textbf{exportieren}, wie in Abbildung~\ref{fig:excelübertragung} schematisch dargestellt.
    \begin{figure}[H]
    \centering
    \includegraphics[width=6cm]{images/Kapitel4/excelübertragung.png}
    \caption{Schematische Übertragung von erkannten Objektdaten in die Ziel-Tabelle}
    \label{fig:excelübertragung}
    \end{figure}

    \item \textbf{FA-010}\label{req:FA-010} \textbf{Strukturerhalt:} Wenn Daten in bestehende Excel-Dateien exportiert werden, \textbf{soll} der Prototyp die vorhandene Struktur (Formatierung, Formeln, Makros) \textbf{bewahren} und ausschließlich Werte in definierte Bereiche einfügen.

    \item \textbf{FA-011}\label{req:FA-011} \textbf{Änderungsverfolgung:} Wenn zwei Planversionen verglichen werden, \textbf{muss} der Prototyp Änderungen \textbf{identifizieren} und \textbf{kategorisieren} (siehe Abbildung~\ref{fig:änderungsample}).
    \begin{figure}[H]
    \centering
    \includegraphics[width=9cm]{images/Kapitel4/änderungsample.png}
    \caption{Visualisierung der Änderungsverfolgung zwischen zwei Planversionen}
    \label{fig:änderungsample}
    \end{figure}

\end{itemize}

\subsection{Benutzerinteraktion und Konfiguration}
\begin{itemize}
    \item \textbf{FA-012}\label{req:FA-012} \textbf{Visuelle Validierung:} Der Prototyp \textbf{soll} alle erkannten Objekte durch farbige Bounding Boxes im Gleisplan oder als Overlay zur manuellen Überprüfung \textbf{visualisieren}, wie in Abbildung~\ref{fig:gleisplanmarkierung} exemplarisch gezeigt.
    \begin{figure}[H]
    \centering
    \includegraphics[width=14cm]{images/Kapitel4/gleisplanmarkierungv2.png}
    \caption{Visuelle Validierung durch Bounding-Box-Overlays im Gleisplan \cite{railroadstationdrawing}}
    \label{fig:gleisplanmarkierung}
    \end{figure}

    \item \textbf{FA-013}\label{req:FA-013} \textbf{Grafische Benutzeroberfläche:} Der Prototyp \textbf{muss} eine grafische Benutzeroberfläche für PDF-Upload, Analyse-Start und Datenexport ohne erforderliche Kommandozeilen-Kenntnisse \textbf{bereitstellen}.

    \item \textbf{FA-014}\label{req:FA-014} \textbf{Modulare Architektur:} Der Prototyp \textbf{muss} mit modularer Architektur \textbf{gestaltet sein}, die klare Trennung der Verarbeitungsstufen (Objekterkennung, Texterkennung, UI, Export) gewährleistet und Erweiterungen (z.\,B. neue Symbolklassen, CAD-Anbindung) ohne Modifikation der Kernlogik ermöglicht.
\end{itemize}

\section{Nicht-funktionale Anforderungen}
\label{sec:nichtfunktionaleanforderungen}
Die nicht-funktionalen Anforderungen wurden aus drei komplementären Quellen abgeleitet:

\begin{enumerate}
    \item \textbf{Stakeholder-Anforderungen:} Interviews und Abstimmungen mit dem Betreuer sowie Fachkollegen der Siemens Mobility GmbH identifizierten praxisrelevante Rahmenbedingungen wie On-Premise-Verarbeitung (NFA-001), Lizenzkonformität (NFA-002) und Prozessoptimierung (NFA-006).

    \item \textbf{Unternehmensrichtlinien:} Die internen IT-Sicherheits- und Compliance-Vorgaben der Siemens Mobility GmbH definieren verbindliche Anforderungen an Datenschutz, Offline-Betrieb und die Verwendung genehmigter Open-Source-Lizenzen.

    \item \textbf{Domänenanalyse und Stand der Technik:} Aus der Analyse bestehender manueller Arbeitsabläufe (vgl. Kapitel~\ref{chap:einleitung}, Forschungsvorgehen Phase~2) sowie der Evaluation verwandter Arbeiten (Kapitel~\ref{chap:theoretischeundtechnischegrundlagen}) wurden Qualitätsanforderungen wie End-to-End-Genauigkeit (NFA-003), Robustheit (NFA-004) und Ressourceneffizienz (NFA-007) abgeleitet.
\end{enumerate}

Ergänzend zu den funktionalen Zielen definieren die nicht-funktionalen Anforderungen (NFA-001 bis NFA-010) die Qualitätsmerkmale und Rahmenbedingungen des Prototyps.

\subsection{Sicherheit und Datenschutz}
\begin{itemize}
    \item \textbf{NFA-001}\label{req:NFA-001} \textbf{On-Premise-Verarbeitung:} Der Prototyp \textbf{muss} alle Daten vollständig lokal ohne Übertragung an externe Cloud-Services \textbf{verarbeiten} und Offline-Betrieb \textbf{unterstützen}.

    \item \textbf{NFA-002}\label{req:NFA-002} \textbf{Lizenzkonformität:} Der Prototyp \textbf{muss} ausschließlich Open-Source-Bibliotheken mit genehmigten Lizenzen (Apache 2.0, MIT, BSD) \textbf{verwenden} und alle Abhängigkeiten \textbf{dokumentieren}.
\end{itemize}

\subsection{Qualität und Zuverlässigkeit}
\begin{itemize}
    \item \textbf{NFA-003}\label{req:NFA-003} \textbf{End-to-End-Genauigkeit:} Der Prototyp \textbf{muss} eine Gesamtgenauigkeit von mindestens 85\,\% bei vollständiger Objektextraktion (Detektion + OCR + Linking + Export korrekt) \textbf{erreichen}.

    \item \textbf{NFA-004}\label{req:NFA-004} \textbf{Robustheit:} Wenn fehlerhafte Eingaben vorliegen (korrupte PDFs, Bildrauschen), \textbf{soll} der Prototyp definierte Fehlermeldungen statt Absturz \textbf{bereitstellen}.

    \item \textbf{NFA-005}\label{req:NFA-005} \textbf{Prüfbarkeit:} Der Prototyp \textbf{soll} visuelle Validierung aller Ergebnisse mittels Bounding-Box-Overlay und Rückverfolgbarkeit (Klick auf Tabellenzeile zeigt Position im Plan) \textbf{bereitstellen}.
\end{itemize}

\subsection{Effizienz und Wirtschaftlichkeit}
\begin{itemize}
    \item \textbf{NFA-006}\label{req:NFA-006} \textbf{Prozessoptimierung:} Der Prototyp \textbf{soll} den manuellen Prüfaufwand durch Transformation vom 4-Augen-Prinzip hin zu einem KI-gestützten Prozess (Mensch prüft KI) \textbf{reduzieren}.

    \item \textbf{NFA-007}\label{req:NFA-007} \textbf{Ressourceneffizienz:} Der Prototyp \textbf{soll} einen durchschnittlichen Bahnhofsplan auf Standard-Hardware (CPU-only) in weniger als der Hälfte der manuellen Bearbeitungszeit \textbf{verarbeiten}.
\end{itemize}

\subsection{Wartbarkeit und Erweiterbarkeit}
\begin{itemize}
    \item \textbf{NFA-008}\label{req:NFA-008} \textbf{Update-Fähigkeit:} Der Prototyp \textbf{soll} die Integration neuer Symbolvarianten über Konfigurations-Updates oder neue Modell-Gewichte ohne Code-Änderungen \textbf{unterstützen}.
\end{itemize}

\subsection{Datenformate und Schnittstellen}
\label{subsec:datenformate}

\begin{itemize}
    \item \textbf{NFA-009}\label{req:NFA-009} \textbf{Eingabeformate:} Der Prototyp \textbf{muss} PDF-Dateien (Vektor/Raster, intern gerendert bei 500 DPI) und Bilddateien (PNG, JPEG, TIFF, BMP mit nativer Auflösung von 500 DPI) \textbf{verarbeiten}.

    \textit{Begründung:} Das YOLOv8-OBB Modell wurde ausschließlich auf 500 DPI Bildkacheln trainiert. Abweichungen führen zu signifikant reduzierten Erkennungsraten.

    \item \textbf{NFA-010}\label{req:NFA-010} \textbf{Ausgabeformate:} Ergänzend zum Excel-Export (FA-009) \textbf{soll} der Prototyp Ergebnisse auch in CSV und JSON Format \textbf{exportieren}.
\end{itemize}

Tabelle \ref{tab:formate-prototyp} gibt einen detaillierten Überblick über die im Prototyp verwendeten Dateiformate und deren Einsatz in der Verarbeitungspipeline.

\begin{table}[H]
\centering
\caption{Übersicht der unterstützten Datenformate (siehe \ref{req:NFA-009} und \ref{req:NFA-010})}
\label{tab:formate-prototyp}
\renewcommand{\arraystretch}{1.4}
\begin{tabularx}{\textwidth}{|p{2.5cm}|p{2cm}|X|X|}
\hline
\textbf{Kategorie} & \textbf{Format} & \textbf{Beschreibung} & \textbf{Einsatz im Prototyp} \\
\hline
\textbf{Eingabe} & PDF & Standardformat für Pläne & Primäre Datenquelle \\
 & PNG/JPG & Rasterisierte Ausschnitte & Input für CNN/YOLO \\
\hline
\textbf{Verarbeitung} & JSON & Strukturierte Metadaten & Interner Datenaustausch \\
 & PostgreSQL & Relationale Datenbank & Persistenz \& Versionierung \\
\hline
\textbf{Ausgabe} & XLSX & Excel-Arbeitsmappe & Engineering-Workflow \\
 & CSV & Textbasiertes Format & Einfacher Datenaustausch \\
 & JSON & API-Response & Schnittstellenanbindung \\
\hline
\end{tabularx}
\end{table}

\section{Herausforderungen bei der Umsetzung}
Die Realisierung der in den Abschnitten \ref{subsec:req_symbol} bis \ref{subsec:datenformate} definierten Anforderungen sieht sich folgenden technischen Herausforderungen gegenüber:

\begin{enumerate}
    \item \textbf{Daten-Heterogenität:} Die Varianz in den Eingabedaten (unterschiedliche Export-Einstellungen, Linienstärken, Skalierungen) erschwert eine universelle Regelbildung.
    \item \textbf{Visuelle Ambiguität:} Einige Symbole (z.\,B. unterschiedliche Gleiskoppelspulen-Typen) unterscheiden sich visuell nur in wenigen Pixeln oder sind nur durch den Kontext (Begleittext) differenzierbar.
    \item \textbf{OCR-Komplexität:} Technischer Text in Plänen ist oft extrem klein, rotiert und durch Führungslinien durchgestrichen, was klassische OCR-Engines (wie Tesseract) an ihre Grenzen bringt.
    \item \textbf{Mangel an Trainingsdaten:} Es existiert kein öffentlicher Datensatz für bahntechnische Symbolik. Ein \enquote{Cold Start} ist notwendig, bei dem Trainingsdaten zunächst manuell (z.\,B. via CVAT) annotiert werden müssen.
    \item \textbf{Semantische Lücke:} Der Schritt von der Erkennung (\enquote{Da ist eine Box}) zur Bedeutung (\enquote{Das ist Weiche 12 in Rechtslage}) erfordert komplexe Heuristiken, insbesondere beim Mapping von Textboxen zu den geometrisch nächsten Symbolen.
\end{enumerate}

Die nachfolgenden Kapitel adressieren diese Herausforderungen durch gezielte Maßnahmen:

\begin{itemize}
    \item \textbf{Daten-Heterogenität:} Synthetische Datenaugmentation durch
    Rotation (10 Winkel von $-90^\circ$ bis $+90^\circ$) sowie Transfer Learning
    von vortrainierten DOTA-Gewichten erhöhen die Robustheit gegenüber
    Eingabevarianz (Kapitel~\ref{chap:implementierung},
    Abschnitt~\ref{subsec:Datensatzerstellung}).

    \item \textbf{Visuelle Ambiguität:} Die Kombination aus OBB-basierter
    Detektion mit YOLOv8 und OCR-gestützter Kontextanalyse ermöglicht die
    Disambiguierung visuell ähnlicher Symbole. Der erkannte Begleittext
    (z.\,B. \enquote{GKS1} vs. \enquote{GKS2}) liefert die entscheidende
    Differenzierungsinformation (Kapitel~\ref{chap:konzeption} und
    \ref{chap:implementierung}).

    \item \textbf{OCR-Komplexität:} Eine Multi-Engine-Kaskade (PaddleOCR,
    Tesseract, EasyOCR) mit orientierungsadaptiver Vorverarbeitung adressiert
    rotierte Texte. Morphologische Operationen entfernen störende Linienelemente
    vor der Erkennung (Kapitel~\ref{chap:theoretischeundtechnischegrundlagen},
    Abschnitt~\ref{subsec:ocr_challenges} sowie Kapitel~\ref{chap:implementierung},
    Abschnitt~\ref{sec:ocrpipeline}).

    \item \textbf{Mangel an Trainingsdaten:} Manuelle Annotation von 24
    Gleisplänen mittels CVAT, erweitert durch synthetische Augmentation
    auf 1.131 Trainingsbilder. Transfer Learning von DOTA-vortrainierten
    Modellgewichten reduziert den Datenbedarf zusätzlich
    (Kapitel~\ref{chap:implementierung}, Abschnitt~\ref{subsec:Datensatzerstellung}).

    \item \textbf{Semantische Lücke:} Ein regelbasiertes Post-Processing-Modul
    mit rotationsinvarianter Koordinatentransformation und klassenspezifischen
    Suchheuristiken überbrückt die Lücke zwischen geometrischer Detektion und
    semantischer Interpretation (Kapitel~\ref{chap:implementierung},
    Abschnitt~\ref{sec:intelligentesymboltextverknüpfung}).
\end{itemize}

\section{Anforderungs-Rückverfolgbarkeit}
\label{sec:traceability}

Die vollständige Rückverfolgbarkeitsmatrix, welche die Zuordnung jeder Anforderung zu den entsprechenden Implementierungskomponenten sowie den Evaluationsmetriken dokumentiert, wird in Kapitel~\ref{chap:evaluation}, Abschnitt~\ref{sec:traceability_eval} präsentiert.