\chapter{Abstract}
\label{chap:abstract}
Railway infrastructure planning relies heavily on technical layout documents in PDF or image format, 
which contain complex arrangements of symbols, text annotations, and positional information. 
Manual extraction and interpretation of this information is time consuming, error prone, and 
difficult to scale across projects and customers. This thesis presents the design and 
prototypical implementation of an intelligent, modular pipeline for automated analysis of 
railway track layouts, developed in cooperation with Siemens Mobility GmbH. The system 
employs a novel dual-angle processing strategy that automatically routes text through 
orientation-specific OCR pipelines: axis-aligned text (±15° from cardinal directions) 
undergoes simple rotation with four-direction sweep testing, while tilted text (>15°) is 
processed using perspective transformation with multi-angle recognition. A multi-engine 
orchestration layer (PaddleOCR primary, Tesseract secondary, EasyOCR fallback) ensures 
robust character recognition across varying document qualities, achieving 92\% accuracy on 
real-world layouts. Deep learning-based object detection using YOLOv8-OBB identifies 
infrastructure elements including signals, switches, track circuits, and coupling coils, 
with rotation-invariant bounding boxes handling symbols at arbitrary orientations. An 
adaptive spatial linking algorithm with learned pattern recognition automatically associates 
detected elements with their textual labels, while rule-based logic determines spatial 
relationships such as driving direction (Fahrtrichtung) and element groupings. All extracted 
objects and metadata are persisted in a PostgreSQL backend employing JSONB storage, 
versioned records, and timestamped deltas to provide built-in change management and 
fine-grained traceability between layout versions. A YAML-based configuration layer enables 
customer-specific symbol mapping, semantic interpretation, and structured export to Excel 
with template support. The solution is evaluated on production layout samples from multiple 
railway projects and demonstrates 95\% detection accuracy for standard symbols, 92\% OCR 
accuracy across all text types, and reduces manual processing time from 4-6 hours to 
approximately 30 minutes per layout. A comprehensive PyQt5-based desktop application 
provides multi-workspace management, interactive graphics with zoom and pan, manual 
correction tools including rotated-rectangle OCR, and visual validation overlays. This work 
contributes to the digitalization of railway infrastructure planning by enabling scalable, 
automated layout analysis that significantly reduces manual effort while improving data 
integrity, auditability, and decision-making efficiency in safety critical railway systems.