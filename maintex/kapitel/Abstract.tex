\chapter{Abstract}
\label{chap:abstract}
Railway infrastructure planning relies heavily on technical layout documents in PDF or image format,
which contain complex arrangements of symbols, text annotations, and positional information.
Manual extraction and interpretation of this information is time-consuming, error-prone, and
difficult to scale across projects and customers. This thesis presents the design and
prototypical implementation of an intelligent, modular pipeline for automated analysis of
railway track layouts (Gleispläne), developed in cooperation with Siemens Mobility GmbH.
The system combines deep learning-based object detection using YOLOv8 with Oriented Bounding
Boxes (OBB) for rotation-invariant symbol recognition, a multi-engine OCR cascade
(PaddleOCR, Tesseract, EasyOCR) with orientation-adaptive preprocessing for robust text
extraction, and an adaptive spatial linking algorithm that automatically associates detected
symbols with their textual labels. Rule-based logic determines spatial relationships such as
driving direction (Fahrtrichtung) and element groupings. All extracted objects and metadata
are persisted in a PostgreSQL backend with JSONB storage, versioned records, and change
tracking for full traceability between layout versions.
The solution is evaluated on nine production A0 layouts from Siemens Mobility projects,
achieving \textbf{97.22\% end-to-end accuracy} for all 12 symbol classes,
integrating detection, OCR, and spatial linking in a unified evaluation metric.
The system reduces manual processing time by \textbf{89.7\%}, from an average of 102 minutes
to approximately 10.6 minutes per layout. A comprehensive PyQt5-based desktop application
provides interactive visualization, manual correction tools, and validation overlays for
human-in-the-loop quality assurance.
This work contributes to the digitalization of railway infrastructure planning by enabling
scalable, automated layout analysis that significantly reduces manual effort while improving
data integrity and decision-making efficiency in safety-critical railway systems.

\vspace{1em}
\noindent\textbf{Keywords:} Deep Learning, Object Detection, YOLOv8-OBB, Optical Character Recognition,
Railway Infrastructure, Document Analysis, Automated Data Extraction
